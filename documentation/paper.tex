\documentclass{article}
\usepackage{neurips_2022} % for first submission 
% \usepackage[preprint]{neurips_2022}
% \usepackage[final]{neurips_2022}
\usepackage[utf8]{inputenc} % allow utf-8 input
\usepackage[T1]{fontenc}    % use 8-bit T1 fonts
\usepackage{hyperref}       % hyperlinks
\usepackage{url}            % simple URL typesetting
\usepackage{booktabs}       % professional-quality tables
\usepackage{amsfonts}       % blackboard math symbols
\usepackage{nicefrac}       % compact symbols for 1/2, etc.
\usepackage{microtype}      % microtypography
\usepackage{xcolor}         % colors
% custom
\usepackage[ruled,vlined,linesnumbered]{algorithm2e}


\title{A Framework for Predictable Actor-Critic Control}
\author{%
    % Josiah Coad, \textsuperscript{\rm 1}
    % Jeff Hykin, \textsuperscript{\rm 2}
    % James Ault, \textsuperscript{\rm 2}
    % Guni Sharon \textsuperscript{\rm 2}
    Jeff Hykin
    \texttt{hippo@cs.cranberry-lemon.edu} \\
    % examples of more authors
    % \And
    % Coauthor \\
    % Affiliation \\
    % Address \\
    % \texttt{email} \\
    % \AND
    % Coauthor \\
    % Affiliation \\
    % Address \\
    % \texttt{email} \\
    % \And
    % Coauthor \\
    % Affiliation \\
    % Address \\
    % \texttt{email} \\
    % \And
    % Coauthor \\
    % Affiliation \\
    % Address \\
    % \texttt{email} \\
}

\newcommand{\AND }{ \land  }
\newcommand{\OR  }{ \lor   }
\newcommand{\NOT }{ \lnot  }
\newcommand{\SUM }{ \Sigma }
\newcommand{\MULT}{ \cdot  }

\begin{document}
    \maketitle
    \begin{abstract}
        Everyday life would be quite difficult without some form of planning and commitment. Simple tasks such as walking across a street often involve choosing predictability above a small but immediate gratification. Current reinforcement learning (RL) techniques often fail to capitalize on the benefits of planning and commitment. Even when agents do plan, they almost always choose the best action at every time-step. This is behavior is followed even if the benefit of abandoning the plan was minuscule. This paper proposes a method allowing an agent follow a plan so long as there is minimal impact on total reward. The work includes both theoretical bounds on the reward impact and supporting experimental data.
    \end{abstract}


    % TODO: refernce format: Citations may be author/year or numeric. Use natbib \url{http://mirrors.ctan.org/macros/latex/contrib/natbib/natnotes.pdf}

    % for graphics:
    %    \usepackage[pdftex]{graphicx} ...
    %    \includegraphics[width=0.8\linewidth]{myfile.pdf}
    % (\url{http://mirrors.ctan.org/macros/latex/required/graphics/grfguide.pdf})


    % 
    % 
    % 
    % Acknowledgments
    % 
    % 
    % 
        % \begin{ack}
        %     FIXME
        %     Do {\bf not} include this section in the anonymized submission, only in the final paper.
            
        %     Use unnumbered first level headings for the acknowledgments.
        %     All acknowledgments go at the end of the paper before the list of references.
        %     - declare funding (financial activities supporting the submitted work)
        %     - competing interests (related financial activities outside the submitted work)
        % \end{ack}


    % 
    % 
    % 
    % References
    % 
    % 
    % 
    \section*{References}
        Use unnumbered first-level heading for references.
        Any choice of citation style is acceptable as long as you are consistent.
        Reference section does not count towards the page limit.
    \medskip





    % 
    % 
    % 
    % Checklist
    % 
    % 
    % 
    \section*{Checklist}
    % change the default \answerTODO{} to \answerYes{}, \answerNo{}, or \answerNA{}
    % examples:
    % \begin{itemize}
    %   \item Did you include the license to the code and datasets? \answerYes{See Section~\ref{gen_inst}.}
    %   \item Did you include the license to the code and datasets? \answerNo{The code and the data are proprietary.}
    %   \item Did you include the license to the code and datasets? \answerNA{}
    % \end{itemize}
    \begin{enumerate}


        \item For all authors...
            \begin{enumerate}
                \item Do the main claims made in the abstract and introduction accurately reflect the paper's contributions and scope?
                    \answerTODO{}
                \item Did you describe the limitations of your work?
                    \answerTODO{}
                \item Did you discuss any potential negative societal impacts of your work?
                    \answerTODO{}
                \item Have you read the ethics review guidelines and ensured that your paper conforms to them?
                    \answerTODO{}
            \end{enumerate}


        \item If you are including theoretical results...
            \begin{enumerate}
                \item Did you state the full set of assumptions of all theoretical results?
                    \answerTODO{}
                \item Did you include complete proofs of all theoretical results?
                    \answerTODO{}
            \end{enumerate}


        \item If you ran experiments...
            \begin{enumerate}
                    \item Did you include the code, data, and instructions needed to reproduce the main experimental results (either in the supplemental material or as a URL)?
                        \answerTODO{}
                    \item Did you specify all the training details (e.g., data splits, hyperparameters, how they were chosen)?
                        \answerTODO{}
                    \item Did you report error bars (e.g., with respect to the random seed after running experiments multiple times)?
                        \answerTODO{}
                    \item Did you include the total amount of compute and the type of resources used (e.g., type of GPUs, internal cluster, or cloud provider)?
                        \answerTODO{}
            \end{enumerate}


        \item If you are using existing assets (e.g., code, data, models) or curating/releasing new assets...
            \begin{enumerate}
                \item If your work uses existing assets, did you cite the creators?
                    \answerTODO{}
                \item Did you mention the license of the assets?
                    \answerTODO{}
                \item Did you include any new assets either in the supplemental material or as a URL?
                    \answerTODO{}
                \item Did you discuss whether and how consent was obtained from people whose data you're using/curating?
                    \answerTODO{}
                \item Did you discuss whether the data you are using/curating contains personally identifiable information or offensive content?
                    \answerTODO{}
            \end{enumerate}


        \item If you used crowdsourcing or conducted research with human subjects...
            \begin{enumerate}
                \item Did you include the full text of instructions given to participants and screenshots, if applicable?
                    \answerTODO{}
                \item Did you describe any potential participant risks, with links to Institutional Review Board (IRB) approvals, if applicable?
                    \answerTODO{}
                \item Did you include the estimated hourly wage paid to participants and the total amount spent on participant compensation?
                    \answerTODO{}
            \end{enumerate}


    \end{enumerate}


    % 
    % 
    % Appendix
    % 
    % 
    \appendix
    \section{Appendix}
    
        % 
        % epsilon optimizer
        % 
        \begin{algorithm}[ht]
            
            \newcommand{\planlensPerEpsilon}{planlens_\epsilon}
            \newcommand{\discountedRewards}{discountedRewards}
            \newcommand{\episodeRewards}{e_r}
            \newcommand{\minimumAcceptableReward}{b_{lower}}
            \newcommand{\numberOfbaselineSamples}{b_{sz}}
            \newcommand{\initialEpsilon}{\epsilon_{i}}
            \newcommand{\runningEpsilon}{\epsilon_{r}}
            \newcommand{\finalEpsilon}{\epsilon_{o}}
            \newcommand{\allEpsilons}{a_\epsilon}
            \newcommand{\intialHorizon}{h_{i}}
            \newcommand{\finalHorizon}{h_{o}}
            \newcommand{\planLengths}{planlens}
            
            % \SetAlgoLined\
            \SetKwInOut{Input}{Input}
            \SetKwInOut{Output}{Output}
            \Input{ \\
                $\mathrm{\minimumAcceptableReward}$, minimum acceptable reward \\
                $\mathrm{iter}$ , number of iterations of refinement \\
                $\mathrm{inc}$, scale multiplier for epsilon \\
                $\mathrm{\numberOfbaselineSamples}$, number of baseline samples \\
                $\mathrm{b_{min}}$, min confidence interval of baseline \\
                $\mathrm{b_{ci}}$, baseline confidence interval width \\
                $\mathrm{ci_{lvl}}$, confidence level  \\
                $\mathrm{\initialEpsilon}$, initial epsilon \\
                $\mathrm{\intialHorizon}$, initial horizon \\
            }
            \Output{
                \\
                $\mathrm{\finalEpsilon}$, a refined epsilon\\
                $\mathrm{\finalHorizon}$, a refined horizon\\
            }
            \caption{Find Epsilon}
            \label{algo:find_epsilon}
            \vspace{8pt}
            
            
            \(\runningEpsilon \gets  \initialEpsilon \)                                                           \\
            \(\allEpsilons \gets []\) \# empty array                                                              \\
            \(\planlensPerEpsilon \gets \{\}\) \# empty hashmap                                                   \\
            \# Note: requesting a non-existent key for this hash-map returns an array containing $\intialHorizon$ \\
            \For{$i$ in $[0,..., iter]$}{
                % \label{ln:loopplan1}
                \(\episodeRewards \gets []\) \# empty array \\
                \For{$j$ in $[0,..., \numberOfbaselineSamples]$}{
                    % \label{ln:loopsample1}
                    \(  \discountedRewards, \planLengths \gets runEpisode(\epsilon=\runningEpsilon) \)                      \\
                    $\episodeRewards.push(\SUM(\discountedRewards))$                                                        \\
                    $\planlensPerEpsilon[\runningEpsilon] \gets concat(\planlensPerEpsilon[\runningEpsilon], \planLengths)$ \\
                    \If{$length(\episodeRewards) < 2$} {
                        $continue$
                    }
                    
                    $s_{max}, s_{min} \gets confidenceInterval(ci_{lvl}, \episodeRewards)$ \\
                    $s_{ci} \gets s_{max} - s_{min}$ \\
                    \If{$ (s_{max} < \minimumAcceptableReward) \OR (s_{ci} < b_{ci}) $}{
                        $break$
                    }
                }
                
                \If{$mean(\episodeRewards) \geq \minimumAcceptableReward$}{
                    $\runningEpsilon \gets \runningEpsilon \MULT inc $
                }
                \Else{
                    $\runningEpsilon \gets \runningEpsilon / inc $
                }
                
                $\allEpsilons.push(\runningEpsilon)$
            }
            $\finalEpsilon \gets median(\allEpsilons)$

            $\finalHorizon \gets max(median(\planlensPerEpsilon[\finalEpsilon]), 1) \MULT 2$

            \Return{
                $\mathrm{\finalEpsilon}, \mathrm{\finalHorizon}$
            }
        \end{algorithm}


\end{document}